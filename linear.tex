\ifx\pdfoutput\undefined\relax\else      % If using pdfTeX
\pdfmapline{cmbx36 cmbx36 <sfbx3583.pfb} % Use cm-super font for cmbx36
\pdfmapline{cmmi10 cmmi10 <cmmi10.pfb}
\pdfmapline{cmmi17 cmmi17 <cmmi12.pfb}   % Replace cmmi17 by cmmi12
\pdfmapline{cmr7 cmr7 <cmr7.pfb}
\pdfmapline{cmr10 cmr10 <cmr10.pfb}
\pdfmapline{cmr17 cmr17 <cmr17.pfb}
\pdfmapline{cmsy10 cmsy10 <cmsy10.pfb}
\pdfmapline{cmsy17 cmsy17 <cmsy10.pfb}   % Replace cmsy17 by cmsy10
\pdfpagewidth=297mm
\pdfpageheight=210mm
\pdfhorigin=0mm
\pdfvorigin=0mm\fi
\hsize=297mm
\vsize=210mm
\nopagenumbers
\font\tit=cmbx36 at 64pt
\font\desc=cmr17 at 17pt
\font\descmath=cmmi17 at 17pt
\font\descsy=cmsy17 at 17pt
\hbox{}
\vfill
\centerline{\tit The Linear Formula}
\vskip 25mm
$$ x = {-b \over a} $$
\vskip 25mm
\centerline{\desc
\textfont0=\desc
\textfont1=\descmath
\textfont2=\descsy
The linear formula gives the solution of $ax+b=0$ for real numbers $a$, $b$ with $a\neq0$.}
\bye
